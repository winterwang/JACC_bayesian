%  LaTeX support: latex@mdpi.com
%DIF LATEXDIFF DIFFERENCE FILE
%DIF DEL Paperfolder_v0_0_0.tex   Thu Jul 16 10:50:49 2020
%DIF ADD Paperfolder.tex          Thu Jul 16 10:51:01 2020
%  In case you need support, please attach all files that are necessary for compiling as well as the log file, and specify the details of your LaTeX setup (which operating system and LaTeX version / tools you are using).

%=================================================================
\documentclass[nutrients,article,submitted,moreauthors,pdftex]{mdpi}

% If you would like to post an early version of this manuscript as a preprint, you may use preprint as the journal and change 'submit' to 'accept'. The document class line would be, e.g., \documentclass[preprints,article,accept,moreauthors,pdftex]{mdpi}. This is especially recommended for submission to arXiv, where line numbers should be removed before posting. For preprints.org, the editorial staff will make this change immediately prior to posting.

%% Some pieces required from the pandoc template
\providecommand{\tightlist}{%
  \setlength{\itemsep}{0pt}\setlength{\parskip}{4pt}}
\setlist[itemize]{leftmargin=*,labelsep=5.8mm}
\setlist[enumerate]{leftmargin=*,labelsep=4.9mm}

\usepackage{longtable}

% see https://stackoverflow.com/a/47122900

%--------------------
% Class Options:
%--------------------
%----------
% journal
%----------
% Choose between the following MDPI journals:
% acoustics, actuators, addictions, admsci, aerospace, agriculture, agriengineering, agronomy, algorithms, animals, antibiotics, antibodies, antioxidants, applsci, arts, asc, asi, atmosphere, atoms, axioms, batteries, bdcc, behavsci , beverages, bioengineering, biology, biomedicines, biomimetics, biomolecules, biosensors, brainsci , buildings, cancers, carbon , catalysts, cells, ceramics, challenges, chemengineering, chemistry, chemosensors, children, cleantechnol, climate, clockssleep, cmd, coatings, colloids, computation, computers, condensedmatter, cosmetics, cryptography, crystals, dairy, data, dentistry, designs , diagnostics, diseases, diversity, drones, econometrics, economies, education, electrochem, electronics, energies, entropy, environments, epigenomes, est, fermentation, fibers, fire, fishes, fluids, foods, forecasting, forests, fractalfract, futureinternet, futurephys, galaxies, games, gastrointestdisord, gels, genealogy, genes, geohazards, geosciences, geriatrics, hazardousmatters, healthcare, heritage, highthroughput, horticulturae, humanities, hydrology, ijerph, ijfs, ijgi, ijms, ijns, ijtpp, informatics, information, infrastructures, inorganics, insects, instruments, inventions, iot, j, jcdd, jcm, jcp, jcs, jdb, jfb, jfmk, jimaging, jintelligence, jlpea, jmmp, jmse, jnt, jof, joitmc, jpm, jrfm, jsan, land, languages, laws, life, literature, logistics, lubricants, machines, magnetochemistry, make, marinedrugs, materials, mathematics, mca, medicina, medicines, medsci, membranes, metabolites, metals, microarrays, micromachines, microorganisms, minerals, modelling, molbank, molecules, mps, mti, nanomaterials, ncrna, neuroglia, nitrogen, notspecified, nutrients, ohbm, particles, pathogens, pharmaceuticals, pharmaceutics, pharmacy, philosophies, photonics, physics, plants, plasma, polymers, polysaccharides, preprints , proceedings, processes, proteomes, psych, publications, quantumrep, quaternary, qubs, reactions, recycling, religions, remotesensing, reports, resources, risks, robotics, safety, sci, scipharm, sensors, separations, sexes, signals, sinusitis, smartcities, sna, societies, socsci, soilsystems, sports, standards, stats, surfaces, surgeries, sustainability, symmetry, systems, technologies, test, toxics, toxins, tropicalmed, universe, urbansci, vaccines, vehicles, vetsci, vibration, viruses, vision, water, wem, wevj

%---------
% article
%---------
% The default type of manuscript is "article", but can be replaced by:
% abstract, addendum, article, benchmark, book, bookreview, briefreport, casereport, changes, comment, commentary, communication, conceptpaper, conferenceproceedings, correction, conferencereport, expressionofconcern, extendedabstract, meetingreport, creative, datadescriptor, discussion, editorial, essay, erratum, hypothesis, interestingimages, letter, meetingreport, newbookreceived, obituary, opinion, projectreport, reply, retraction, review, perspective, protocol, shortnote, supfile, technicalnote, viewpoint
% supfile = supplementary materials

%----------
% submit
%----------
% The class option "submit" will be changed to "accept" by the Editorial Office when the paper is accepted. This will only make changes to the frontpage (e.g., the logo of the journal will get visible), the headings, and the copyright information. Also, line numbering will be removed. Journal info and pagination for accepted papers will also be assigned by the Editorial Office.

%------------------
% moreauthors
%------------------
% If there is only one author the class option oneauthor should be used. Otherwise use the class option moreauthors.

%---------
% pdftex
%---------
% The option pdftex is for use with pdfLaTeX. If eps figures are used, remove the option pdftex and use LaTeX and dvi2pdf.

%=================================================================
\firstpage{1}
\makeatletter
\setcounter{page}{\@firstpage}
\makeatother
\pubvolume{xx}
\issuenum{1}
\articlenumber{5}
\pubyear{2019}
\copyrightyear{2019}
%\externaleditor{Academic Editor: name}
\history{Received: date; Accepted: date; Published: date}
\updates{yes} % If there is an update available, un-comment this line

%% MDPI internal command: uncomment if new journal that already uses continuous page numbers
%\continuouspages{yes}

%------------------------------------------------------------------
% The following line should be uncommented if the LaTeX file is uploaded to arXiv.org
%\pdfoutput=1

%=================================================================
% Add packages and commands here. The following packages are loaded in our class file: fontenc, calc, indentfirst, fancyhdr, graphicx, lastpage, ifthen, lineno, float, amsmath, setspace, enumitem, mathpazo, booktabs, titlesec, etoolbox, amsthm, hyphenat, natbib, hyperref, footmisc, geometry, caption, url, mdframed, tabto, soul, multirow, microtype, tikz

%=================================================================
%% Please use the following mathematics environments: Theorem, Lemma, Corollary, Proposition, Characterization, Property, Problem, Example, ExamplesandDefinitions, Hypothesis, Remark, Definition
%% For proofs, please use the proof environment (the amsthm package is loaded by the MDPI class).

%=================================================================
% Full title of the paper (Capitalized)
\Title{Milk intake and stroke mortality in the Japan Collaborative
Cohort Study - a Bayesian survival analysis}

% Authors, for the paper (add full first names)
\Author{Chaochen
Wang$^{1,*}$\href{https://orcid.org/0000-0001-5533-1497}{\orcidicon}, Hiroshi
Yatsuya$^{2}$\href{https://orcid.org/0000-0002-6220-9251}{\orcidicon}, Yingsong
Lin$^{1}$\href{https://orcid.org/0000-0003-0214-3649}{\orcidicon}, Tae
Sasakabe$^{1}$\href{https://orcid.org/0000-0002-7417-5693}{\orcidicon}, Sayo
Kawai$^{1}$\href{https://orcid.org/0000-0002-5156-4880}{\orcidicon}, Shogo
Kikuchi$^{1}$\href{https://orcid.org/0000-0001-9499-9668}{\orcidicon}, Hiroyasu
Iso$^{3}$\href{https://orcid.org/0000-0002-9241-7289}{\orcidicon}, Akiko
Tamakoshi$^{4}$\href{https://orcid.org/0000-0002-9761-3879}{\orcidicon}}

% Authors, for metadata in PDF
\AuthorNames{Chaochen Wang, Hiroshi Yatsuya, Yingsong Lin, Tae
Sasakabe, Sayo Kawai, Shogo Kikuchi, Hiroyasu Iso, Akiko Tamakoshi}

% Affiliations / Addresses (Add [1] after \address if there is only one affiliation.)
\address{%
$^{1}$ \quad Department of Public Health, Aichi Medical University
School of Medicine, Nagakute, Japan; \\
$^{2}$ \quad Departmet of Public Health, Fujita Health University School
of Medicine, Toyoake, Japan; \\
$^{3}$ \quad Public Health, Department of Social Medicine, Osaka
University Graduate School of Medicine, Osaka, Japan; \\
$^{4}$ \quad Department of Public Health, Faculty of Medicine, Hokkaido
University, Sapporo, Japan; \\
}
% Contact information of the corresponding author
\corres{Correspondence: Email.:
\href{mailto:chaochen@wangcc.me}{\nolinkurl{chaochen@wangcc.me}}; Tel.:
+81-561-62-3311. Department of Public Health, Aichi Medical University
School of Medicine, 1-1 Yazakokarimata, Nagakute, Aichi, 480-1195, Japan
(C.W.)}

% Current address and/or shared authorship








% The commands \thirdnote{} till \eighthnote{} are available for further notes

% Simple summary

% Abstract (Do not insert blank lines, i.e. \\)
\abstract{The aim was to further examine the relationship between milk
intake and stroke mortality among the Japanese population. We used data
%DIF 132-137c132-137
%DIF < from the Japan Collaborative Cohort Study to estimate the posterior
%DIF < acceleration factors (AF) as well as the hazard ratios (HR) comparing
%DIF < individuals with different milk intake frequencies against those who
%DIF < never consumed milk at the study baseline. These estimations were
%DIF < computed through a series of Bayesian survival models that employed a
%DIF < Markov Chain Monte Carlo simulation process. 100000 posterior samples
%DIF -------
from the Japan Collaborative Cohort (JACC) Study to estimate the %DIF > 
posterior acceleration factors (AF) as well as the hazard ratios (HR) %DIF > 
comparing individuals with different milk intake frequencies against %DIF > 
those who never consumed milk at the study baseline. These estimations %DIF > 
were computed through a series of Bayesian survival models that employed %DIF > 
a Markov Chain Monte Carlo simulation process. 100000 posterior samples %DIF > 
%DIF -------
for each individual were generated separately through four independent
chains after model convergency were confirmed. Posterior probabilites
that daily milk consumers had lower hazard or delayed mortality from
strokes compared to non-consumers was 99.0\% and 78.0\% for men and
women, respectively. Accordingly, the estimated posterior means of AF
and HR for daily milk consumers were 0.88 (95\% Credible Interval, CrI:
0.81, 0.96) and 0.80 (95\% CrI: 0.69, 0.93) for men and 0.97 (95\% CrI:
%DIF 145c145-148
%DIF < 0.88, 1.10) and 0.95 (95\% CrI: 0.80, 1.17) for women.}
%DIF -------
0.88, 1.10) and 0.95 (95\% CrI: 0.80, 1.17) for women. \DIFaddbegin \DIFadd{In conclusion, %DIF > 
data from the JACC study has provided strong evidence that daily milk %DIF > 
intake among Japanese men was associated with delayed and lower hazard %DIF > 
of mortality from stroke especially cerebral infarction. }\DIFaddend } %DIF > 
%DIF -------

% Keywords
\keyword{milk intake; mortality; stroke; Bayesian survival anlysis;
time-to-event data; JACC study}

% The fields PACS, MSC, and JEL may be left empty or commented out if not applicable
%\PACS{J0101}
%\MSC{}
%\JEL{}

%%%%%%%%%%%%%%%%%%%%%%%%%%%%%%%%%%%%%%%%%%
% Only for the journal Diversity
%\LSID{\url{http://}}

%%%%%%%%%%%%%%%%%%%%%%%%%%%%%%%%%%%%%%%%%%
% Only for the journal Applied Sciences:
%\featuredapplication{Authors are encouraged to provide a concise description of the specific application or a potential application of the work. This section is not mandatory.}
%%%%%%%%%%%%%%%%%%%%%%%%%%%%%%%%%%%%%%%%%%

%%%%%%%%%%%%%%%%%%%%%%%%%%%%%%%%%%%%%%%%%%
% Only for the journal Data:
%\dataset{DOI number or link to the deposited data set in cases where the data set is published or set to be published separately. If the data set is submitted and will be published as a supplement to this paper in the journal Data, this field will be filled by the editors of the journal. In this case, please make sure to submit the data set as a supplement when entering your manuscript into our manuscript editorial system.}

%\datasetlicense{license under which the data set is made available (CC0, CC-BY, CC-BY-SA, CC-BY-NC, etc.)}

%%%%%%%%%%%%%%%%%%%%%%%%%%%%%%%%%%%%%%%%%%
% Only for the journal Toxins
%\keycontribution{The breakthroughs or highlights of the manuscript. Authors can write one or two sentences to describe the most important part of the paper.}

%\setcounter{secnumdepth}{4}
%%%%%%%%%%%%%%%%%%%%%%%%%%%%%%%%%%%%%%%%%%


\usepackage{graphicx}
%DIF PREAMBLE EXTENSION ADDED BY LATEXDIFF
%DIF UNDERLINE PREAMBLE %DIF PREAMBLE
\RequirePackage[normalem]{ulem} %DIF PREAMBLE
\RequirePackage{color}\definecolor{RED}{rgb}{1,0,0}\definecolor{BLUE}{rgb}{0,0,1} %DIF PREAMBLE
\providecommand{\DIFadd}[1]{{\protect\color{blue}\uwave{#1}}} %DIF PREAMBLE
\providecommand{\DIFdel}[1]{{\protect\color{red}\sout{#1}}}                      %DIF PREAMBLE
%DIF SAFE PREAMBLE %DIF PREAMBLE
\providecommand{\DIFaddbegin}{} %DIF PREAMBLE
\providecommand{\DIFaddend}{} %DIF PREAMBLE
\providecommand{\DIFdelbegin}{} %DIF PREAMBLE
\providecommand{\DIFdelend}{} %DIF PREAMBLE
\providecommand{\DIFmodbegin}{} %DIF PREAMBLE
\providecommand{\DIFmodend}{} %DIF PREAMBLE
%DIF FLOATSAFE PREAMBLE %DIF PREAMBLE
\providecommand{\DIFaddFL}[1]{\DIFadd{#1}} %DIF PREAMBLE
\providecommand{\DIFdelFL}[1]{\DIFdel{#1}} %DIF PREAMBLE
\providecommand{\DIFaddbeginFL}{} %DIF PREAMBLE
\providecommand{\DIFaddendFL}{} %DIF PREAMBLE
\providecommand{\DIFdelbeginFL}{} %DIF PREAMBLE
\providecommand{\DIFdelendFL}{} %DIF PREAMBLE
\newcommand{\DIFscaledelfig}{0.5}
%DIF HIGHLIGHTGRAPHICS PREAMBLE %DIF PREAMBLE
\RequirePackage{settobox} %DIF PREAMBLE
\RequirePackage{letltxmacro} %DIF PREAMBLE
\newsavebox{\DIFdelgraphicsbox} %DIF PREAMBLE
\newlength{\DIFdelgraphicswidth} %DIF PREAMBLE
\newlength{\DIFdelgraphicsheight} %DIF PREAMBLE
% store original definition of \includegraphics %DIF PREAMBLE
\LetLtxMacro{\DIFOincludegraphics}{\includegraphics} %DIF PREAMBLE
\newcommand{\DIFaddincludegraphics}[2][]{{\color{blue}\fbox{\DIFOincludegraphics[#1]{#2}}}} %DIF PREAMBLE
\newcommand{\DIFdelincludegraphics}[2][]{% %DIF PREAMBLE
\sbox{\DIFdelgraphicsbox}{\DIFOincludegraphics[#1]{#2}}% %DIF PREAMBLE
\settoboxwidth{\DIFdelgraphicswidth}{\DIFdelgraphicsbox} %DIF PREAMBLE
\settoboxtotalheight{\DIFdelgraphicsheight}{\DIFdelgraphicsbox} %DIF PREAMBLE
\scalebox{\DIFscaledelfig}{% %DIF PREAMBLE
\parbox[b]{\DIFdelgraphicswidth}{\usebox{\DIFdelgraphicsbox}\\[-\baselineskip] \rule{\DIFdelgraphicswidth}{0em}}\llap{\resizebox{\DIFdelgraphicswidth}{\DIFdelgraphicsheight}{% %DIF PREAMBLE
\setlength{\unitlength}{\DIFdelgraphicswidth}% %DIF PREAMBLE
\begin{picture}(1,1)% %DIF PREAMBLE
\thicklines\linethickness{2pt} %DIF PREAMBLE
{\color[rgb]{1,0,0}\put(0,0){\framebox(1,1){}}}% %DIF PREAMBLE
{\color[rgb]{1,0,0}\put(0,0){\line( 1,1){1}}}% %DIF PREAMBLE
{\color[rgb]{1,0,0}\put(0,1){\line(1,-1){1}}}% %DIF PREAMBLE
\end{picture}% %DIF PREAMBLE
}\hspace*{3pt}}} %DIF PREAMBLE
} %DIF PREAMBLE
\LetLtxMacro{\DIFOaddbegin}{\DIFaddbegin} %DIF PREAMBLE
\LetLtxMacro{\DIFOaddend}{\DIFaddend} %DIF PREAMBLE
\LetLtxMacro{\DIFOdelbegin}{\DIFdelbegin} %DIF PREAMBLE
\LetLtxMacro{\DIFOdelend}{\DIFdelend} %DIF PREAMBLE
\DeclareRobustCommand{\DIFaddbegin}{\DIFOaddbegin \let\includegraphics\DIFaddincludegraphics} %DIF PREAMBLE
\DeclareRobustCommand{\DIFaddend}{\DIFOaddend \let\includegraphics\DIFOincludegraphics} %DIF PREAMBLE
\DeclareRobustCommand{\DIFdelbegin}{\DIFOdelbegin \let\includegraphics\DIFdelincludegraphics} %DIF PREAMBLE
\DeclareRobustCommand{\DIFdelend}{\DIFOaddend \let\includegraphics\DIFOincludegraphics} %DIF PREAMBLE
\LetLtxMacro{\DIFOaddbeginFL}{\DIFaddbeginFL} %DIF PREAMBLE
\LetLtxMacro{\DIFOaddendFL}{\DIFaddendFL} %DIF PREAMBLE
\LetLtxMacro{\DIFOdelbeginFL}{\DIFdelbeginFL} %DIF PREAMBLE
\LetLtxMacro{\DIFOdelendFL}{\DIFdelendFL} %DIF PREAMBLE
\DeclareRobustCommand{\DIFaddbeginFL}{\DIFOaddbeginFL \let\includegraphics\DIFaddincludegraphics} %DIF PREAMBLE
\DeclareRobustCommand{\DIFaddendFL}{\DIFOaddendFL \let\includegraphics\DIFOincludegraphics} %DIF PREAMBLE
\DeclareRobustCommand{\DIFdelbeginFL}{\DIFOdelbeginFL \let\includegraphics\DIFdelincludegraphics} %DIF PREAMBLE
\DeclareRobustCommand{\DIFdelendFL}{\DIFOaddendFL \let\includegraphics\DIFOincludegraphics} %DIF PREAMBLE
%DIF END PREAMBLE EXTENSION ADDED BY LATEXDIFF

\begin{document}
%%%%%%%%%%%%%%%%%%%%%%%%%%%%%%%%%%%%%%%%%%

\hypertarget{introduction}{%
\section{Introduction}\label{introduction}}

Dairy food, especially milk has been \DIFdelbegin \DIFdel{recommended }\DIFdelend \DIFaddbegin \DIFadd{suggested }\DIFaddend to reduce stroke risk by
nearly 7\% for each 200 g increment of daily consumption
\citep{DeGoede2016}. More intuitive interpretation for a decreasing risk
would be possible if we were able to compute the exact probability for
people who had milk intake may had lower hazard of dying from stroke
compared with those who never drank milk at all. For general
public/media reporting, concept of hazard in epidemiological studies
could still sometimes be challenging to be understood or misinterpreted
since hazard is formally defined as the probability of the occurrence of
an event at a given time point \citep{collett2015modelling}. Usually,
authors of epidemiological papers would tend to use ``risk'' instead of
``hazard'' or interchangeably. However, it would still possibly be mixed
up with ``risk'' that only contain pure meaning of ``probability of an
event'' without redefining a point or a period of time in
cross-sectional settings. For better understanding and interpretation of
the findings from data that researchers endeavored to collect,
statistical literature have provided plenty of choices that could help
us better communicate with each other. Another approach of comparing the
time-to-event survival probabilities between different groups would be
to model the time before observing an event rather than the hazard which
always required the assumption of a proportional hazard to be met.
Accelerated failure time models are among these convenient tools that
would avoid worrying about the assumption of proportional hazard and
directly showing how faster/slower one individual in an exposure group
might have an event compared to others among different exposure groups
\citep{Wei1992}.

Our aim was to overcome these potential pitfalls, avoid
misunderstanding, and provide a more straightforward answer to the main
research question that whether someone answered he/she drank milk at the
baseline of study had lower hazard of dying from stroke compared with
his/her counterparts who said they never consumed milk. If the answer to
the primary objective was yes, then the probabilities that individuals
with different frequencies of milk intake may had lower hazard compared
with those who never drank milk were calculated through a Markov Chain
Monte Carlo (MCMC) simulation process. A Bayesian survival analysis
method was applied on an existing database and through which, we also
provided estimates about whether drinking milk could delay or slow down
the speed towards a mortality from stroke event from happening after
controlling for the other potential confounders.

\hypertarget{materials-and-methods}{%
\section{Materials and Methods}\label{materials-and-methods}}

\hypertarget{the-database}{%
\subsection{The database}\label{the-database}}

We used data from the Japan Collaborative Cohort (JACC) study, which was
sponsored by the Ministry of Education, Sports, Science, and Technology
of Japan. Sampling methods and details about the JACC study have been
described extensively in the literature
\citep{Ohno2001, Tamakoshi2005, Tamakoshi2013}. Participants of the JACC
study completed self-administered questionnaires about their lifestyles,
food intake (food frequency questionnaire, FFQ), and medical histories
of cardiovascular disease or cancer. In the final follow-up of the JACC
study, data from a total of 110585 individuals (46395 men and 64190
women) were successfully retained for the current analysis. We further
excluded samples if they meet one of the following criteria: 1) with any
disease history of stroke, cancer, myocardial infarction, ischemic heart
disease, or other types heart disease (n = 6655, 2931 men and 3724
women); 2) did not answer the question regarding their milk consumption
in the baseline FFQ survey (n = 9545, 3593 men and 5952 women). Finally,
94385 (39386 men and 54999 women) are left in the database. The study
design and informed consent procedure were approved by the Ethics Review
Committee of Nagoya University School of Medicine.

\hypertarget{exposure-and-the-outcome-of-interest}{%
\subsection{Exposure and the outcome of
interest}\label{exposure-and-the-outcome-of-interest}}

Frequency of milk intake during the preceding year of the baseline was
assessed by FFQ from ``never'', ``1-2 times/month'', ``1-2 times/week'',
``3-4 times/week'', and ``Almost daily''. The exact amount of milk
consumption was difficult to assess here. However, good reproducibility
and validity were confirmed previously (Spearman rank correlation
coefficient between milk intake frequency and weighed dietary record for
12 days was 0.65) \citep{Date2005}.

The causes and date of death were obtained from death certificates and
were systematically reviewed. The follow-up period was defined as from
the time of the baseline survey was completed, which was between
1988-1990, until the end of 2009 (administrative censor), or the date
when move-out of study area, or the date of death from stroke recorded,
whichever occurred first. Other causes of death were treated as censored
and assumed not informative. The causes of death were coded by the 10th
Revision of the International Statistical Classification of Diseases and
Related Health Problems (ICD-10), therefore stroke was defined as
I60-I69. We further classified these deaths into hemorrhagic stroke
(I60, I61 and I62) or cerebral infarction (I63) when subtypes of stroke
in their death certificates were available.

\hypertarget{statistical-approach}{%
\subsection{Statistical approach}\label{statistical-approach}}

We calculated sex-specific means (standard deviation, sd) and proportion
of selected baseline characteristics according to the frequency of milk
intake. Overall difference across the milk intake groups were tested by
either analysis of variance for continuous variables or \(\chi^2\) test
for categorical variables.

Full parametric proportional hazard models under Bayesian framework with
Weibull distribution were fitted using Just Another Gibbs Sampler (JAGS)
program \citep{Plummer2003} version 4.3.0 in R version 4.0.1
\citep{RCT2020}. JAGS program is similar to the OpenBUGS
\citep{Lunn2009} project that uses a Gibbs sampling engine for MCMC
simulation. In the current analysis, we specified non-informative prior
distributions for each of the parameters in our models
(\(\beta_n \sim N(0, 1000)\), and
\(\kappa_{\text{shape}} \sim \Gamma(0.001, 0.001)\)). The
Brooks-Gelman-Rubin diagnostic \citep{Brooks1998} was used to refine the
approximate point of convergence, the point when the ratio of the chains
is stable around 1 and the within and between chain variability start to
reach stability was visually checked. The auto-correlation tool further
identified if convergence has been achieved or if a high degree of
auto-correlation exists in the sample. Then, the number of iterations
discarded as `burn-in' was chosen. All models had a posterior sample
size of 100000 from four separated chains with a ``burn-in'' of 2500
iterations. Posterior means (sd) and 95\% Credible Intervals (CrI) of
the estimated hazard ratios (HRs) as well as acceleration factors (AFs)
were presented for each category of milk intake frequency taking the
``never'' category as the reference. Posterior probabilities that the
estimated hazard of dying from stroke for the milk intake for frequency
that higher or equal to ``1-2 times/month'' is smaller compared with
those who chose ``never'' to their milk intake frequency were calculated
as \(P(\text{HR} < 1)\).

The parametric forms of the models fitted in the Bayesian survival
analyses included three models: 1) the crude model, 2) the age-centered
adjusted model, 3) and a model further adjusted for potential
confounders which includes: age (centered, continuous), smoking habit
(never, current, former), alcohol intake (never or past, \(<\) 4
times/week, Daily), body mass index (\(<\) 18.5, \(\geq\) 18.5 and \(<\)
25, \(\geq\) 25 and \(<\) 30, \(\geq\) 30 kg/m\(^2\)), history of
hypertension, diabetes, kidney/liver diseases (yes/no), exercise (more
than 1 hour/week, yes/no), sleep duration (\(<\) 7, \(\geq\) 7 and \(<\)
8, \(\geq\) 8 and \(<\) 9, \(\geq\) 9, hours), coffee intake (never,
\(<\) 3-4 times/week, almost daily), education level (attended school
till age 18, yes/no)

\begin{table}[h]
\DIFaddbeginFL 

\DIFaddendFL \caption{\label{tab:tab1f}Sex-specific baseline characteristics according to the frequency of milk intake  (JACC study, 1988-2009).}
\centering
\fontsize{8}{10}\selectfont
\DIFdelbeginFL %DIFDELCMD < \resizebox{\textwidth}{!}{%
%DIFDELCMD < \begin{tabular}[t]{lccccccc}
%DIFDELCMD < \toprule
%DIFDELCMD < \multicolumn{1}{c}{ } & \multicolumn{1}{c}{} & \multicolumn{1}{c}{} & \multicolumn{4}{c}{\textbf{Milk drinkers}} & \multicolumn{1}{c}{} \\
%DIFDELCMD < \cmidrule(l{3pt}r{3pt}){4-7}
%DIFDELCMD <  & \textbf{Never} & \textbf{Drinkers} & \textbf{1-2 times/} & \textbf{1-2 times/} & \textbf{3-4 times/} & \textbf{Almost} & \\
%DIFDELCMD <  &       &         & \textbf{Month}     & \textbf{Week}      & \textbf{Week}         & \textbf{Daily}  & \textbf{\textit{P} value}\\
%DIFDELCMD < \midrule
%DIFDELCMD < \rowcolor{gray!6}  \addlinespace[0.3em]
%DIFDELCMD < \multicolumn{8}{l}{\textbf{Men (n = 39386)}}\\
%DIFDELCMD < \hspace{1em}number of subjects & 8508 & 30878 & 3522 & 5928 & 5563 & 15865 & \\
%DIFDELCMD < \hspace{1em}Age, year (mean (SD)) & 56.8 (9.9) & 56.8 (10.2) & 55.2 (10.1) & 55.4 (10.1) & 55.4 (9.9) & 58.1 (10.1) & <0.001\\
%DIFDELCMD < \rowcolor{gray!6}  \hspace{1em}Current smoker, \% & 58.7 & 49.8 & 57.4 & 55.9 & 51.1 & 45.4 & <0.001\\
%DIFDELCMD < \hspace{1em}Daily alcohol drinker, \% & 51.9 & 47.8 & 50.9 & 48.4 & 48.6 & 46.5 & <0.001\\
%DIFDELCMD < \rowcolor{gray!6}  \hspace{1em}BMI, kg/m$^2$ (mean (SD)) & 22.6 (3.4) & 22.7 (3.4) & 22.8 (2.8) & 22.8 (2.8) & 22.9 (5.4) & 22.6 (2.8) & <0.001\\
%DIFDELCMD < \hspace{1em}Exercise (> 1h/week), \% & 19.0 & 27.6 & 26.5 & 25.0 & 25.5 & 29.5 & <0.001\\
%DIFDELCMD < \rowcolor{gray!6}  \hspace{1em}Sleep duration, 8-9 hours, \% & 35.6 & 35.9 & 34.6 & 36.2 & 35.1 & 36.3 & <0.001\\
%DIFDELCMD < \hspace{1em}Vegetable intake, daily, \% & 21.3 & 25.4 & 20.1 & 20.4 & 20.8 & 30.1 & <0.001\\
%DIFDELCMD < \rowcolor{gray!6}  \hspace{1em}Fruit intake, daily, \% & 14.8 & 22.4 & 15.4 & 16.3 & 17.3 & 28.1 & <0.001\\
%DIFDELCMD < \hspace{1em}Green tea intake, daily, \% & 76.5 & 79.2 & 79.9 & 78.3 & 77.9 & 79.8 & <0.001\\
%DIFDELCMD < \rowcolor{gray!6}  \hspace{1em}Coffee intake, daily, \% & 43.8 & 50.7 & 50.5 & 48.0 & 47.5 & 52.9 & <0.001\\
%DIFDELCMD < \hspace{1em}Educated over 18 years old, \% & 25.5 & 34.7 & 33.8 & 33.3 & 31.0 & 36.6 & <0.001\\
%DIFDELCMD < \rowcolor{gray!6}  \hspace{1em}History of diabetes, \% & 5.5 & 6.3 & 4.5 & 4.2 & 5.5 & 7.7 & <0.001\\
%DIFDELCMD < \hspace{1em}History of hypertension, \% & 18.4 & 17.9 & 17.5 & 17.1 & 16.8 & 18.7 & 0.039\\
%DIFDELCMD < \rowcolor{gray!6}  \hspace{1em}History of kidney diseases, \% & 3.0 & 3.4 & 3.8 & 3.0 & 3.0 & 3.5 & <0.001\\
%DIFDELCMD < \hspace{1em}History of liver diseases, \% & 5.8 & 6.5 & 6.3 & 6.0 & 5.4 & 7.2 & <0.001\\
%DIFDELCMD < \rowcolor{gray!6}  \addlinespace[0.3em]
%DIFDELCMD < \multicolumn{8}{l}{\textbf{Women (n = 545999)}}\\
%DIFDELCMD < \hspace{1em}number of subjects & 10407 & 44592 & 3640 & 7590 & 8108 & 25254 & \\
%DIFDELCMD < \hspace{1em}Age, year (mean (SD)) & 58.0 (10.2) & 56.9 (9.9) & 56.5 (10.2) & 55.6 (10.1) & 55.6 (9.9) & 57.9 (9.9) & <0.001\\
%DIFDELCMD < \rowcolor{gray!6}  \hspace{1em}Current smoker, \% & 6.9 & 4.2 & 6.1 & 5.5 & 4.3 & 3.5 & <0.001\\
%DIFDELCMD < \hspace{1em}Daily alcohol drinker, \% & 4.3 & 4.5 & 5.5 & 4.3 & 4.2 & 4.6 & <0.001\\
%DIFDELCMD < \rowcolor{gray!6}  \hspace{1em}BMI, kg/m2 (mean (SD)) & 23.0 (3.4) & 22.9 (3.7) & 23.0 (3.8) & 23.1 (4.4) & 23.1 (3.1) & 22.8 (3.6) & <0.001\\
%DIFDELCMD < \hspace{1em}Exercise (> 1h/week), \% & 13.6 & 20.8 & 17.1 & 18.5 & 18.8 & 22.6 & <0.001\\
%DIFDELCMD < \rowcolor{gray!6}  \hspace{1em}Sleep duration, 8-9 hours, \% & 27.7 & 25.6 & 25.1 & 25.9 & 25.4 & 25.7 & <0.001\\
%DIFDELCMD < \hspace{1em}Vegetable intake, daily, \% & 24.7 & 30.4 & 25.0 & 24.6 & 24.2 & 34.8 & <0.001\\
%DIFDELCMD < \rowcolor{gray!6}  \hspace{1em}Fruit intake, daily, \% & 25.0 & 35.7 & 26.6 & 29.2 & 29.2 & 41.1 & <0.001\\
%DIFDELCMD < \hspace{1em}Green tea intake, daily, \% & 73.8 & 76.8 & 77.0 & 76.4 & 75.8 & 77.3 & <0.001\\
%DIFDELCMD < \rowcolor{gray!6}  \hspace{1em}Coffee intake, daily, \% & 39.6 & 48.2 & 46.2 & 46.4 & 44.4 & 50.2 & <0.001\\
%DIFDELCMD < \hspace{1em}Educated over 18 years old, \% & 19.9 & 31.6 & 27.9 & 29.8 & 27.4 & 34.0 & <0.001\\
%DIFDELCMD < \rowcolor{gray!6}  \hspace{1em}History of diabetes, \% & 2.6 & 3.7 & 3.2 & 2.7 & 2.7 & 4.4 & <0.001\\
%DIFDELCMD < \hspace{1em}History of hypertension, \% & 21.5 & 19.7 & 20.5 & 19.1 & 18.9 & 20.0 & <0.001\\
%DIFDELCMD < \rowcolor{gray!6}  \hspace{1em}History of kidney diseases, \% & 3.6 & 4.1 & 3.9 & 3.7 & 3.7 & 4.4 & <0.001\\
%DIFDELCMD < \hspace{1em}History of liver diseases, \% & 3.5 & 4.6 & 4.9 & 3.9 & 3.9 & 5.0 & <0.001\\
%DIFDELCMD < \bottomrule
%DIFDELCMD < \end{tabular}}
%DIFDELCMD < %%%
\DIFdelendFL \DIFaddbeginFL \resizebox{\textwidth}{!}{%
\begin{tabular}[t]{lcccccc}
\toprule
\multicolumn{1}{c}{ } & \multicolumn{1}{c}{} & \multicolumn{1}{c}{} & \multicolumn{4}{c}{\textbf{Milk drinkers}} \\
\cmidrule(l{3pt}r{3pt}){4-7}
 & \textbf{Never} & \textbf{Drinkers} & \textbf{1-2 times/} & \textbf{1-2 times/} & \textbf{3-4 times/} & \textbf{Almost}  \\
 &       &         & \textbf{Month}     & \textbf{Week}      & \textbf{Week}         & \textbf{Daily}  \\
\midrule
\rowcolor{gray!6}  \addlinespace[0.3em]
\multicolumn{7}{l}{\textbf{Men (n = 39386)}}\\
\hspace{1em}Number of subjects & 8508 & 30878 & 3522 & 5928 & 5563 & 15865\\
\hspace{1em}Age, year (mean (SD)) & 56.8 (9.9) & 56.8 (10.2) & 55.2 (10.1) & 55.4 (10.1) & 55.4 (9.9) & 58.1 (10.1)\\
\rowcolor{gray!6}  \hspace{1em}Current smoker, \% & 58.7 & 49.8 & 57.4 & 55.9 & 51.1 & 45.4\\
\hspace{1em}Daily alcohol drinker, \% & 51.9 & 47.8 & 50.9 & 48.4 & 48.6 & 46.5\\
\rowcolor{gray!6}  \hspace{1em}BMI, kg/m$^2$ (mean (SD)) & 22.6 (3.4) & 22.7 (3.4) & 22.8 (2.8) & 22.8 (2.8) & 22.9 (5.4) & 22.6 (2.8)\\
\hspace{1em}Exercise (> 1h/week), \% & 19.0 & 27.6 & 26.5 & 25.0 & 25.5 & 29.5\\
\rowcolor{gray!6}  \hspace{1em}Sleep duration, 8-9 hours, \% & 35.6 & 35.9 & 34.6 & 36.2 & 35.1 & 36.3\\
\hspace{1em}Vegetable intake, daily, \% & 21.3 & 25.4 & 20.1 & 20.4 & 20.8 & 30.1\\
\rowcolor{gray!6}  \hspace{1em}Fruit intake, daily, \% & 14.8 & 22.4 & 15.4 & 16.3 & 17.3 & 28.1\\
\hspace{1em}Green tea intake, daily, \% & 76.5 & 79.2 & 79.9 & 78.3 & 77.9 & 79.8\\
\rowcolor{gray!6}  \hspace{1em}Coffee intake, daily, \% & 43.8 & 50.7 & 50.5 & 48.0 & 47.5 & 52.9\\
\hspace{1em}Educated over 18 years old, \% & 25.5 & 34.7 & 33.8 & 33.3 & 31.0 & 36.6\\
\rowcolor{gray!6}  \hspace{1em}History of diabetes, \% & 5.0 & 6.3 & 4.5 & 4.2 & 5.5 & 7.7\\
\hspace{1em}History of hypertension, \% & 18.4 & 17.9 & 17.5 & 17.1 & 16.8 & 18.7\\
\rowcolor{gray!6}  \hspace{1em}History of kidney diseases, \% & 3.0 & 3.4 & 3.8 & 3.0 & 3.0 & 3.5\\
\hspace{1em}History of liver diseases, \% & 5.8 & 6.5 & 6.3 & 6.0 & 5.4 & 7.2\\
\rowcolor{gray!6}  \addlinespace[0.3em]
\multicolumn{7}{l}{\textbf{Women (n = 54999)}}\\
\hspace{1em}number of subjects & 10407 & 44592 & 3640 & 7590 & 8108 & 25254\\
\hspace{1em}Age, year (mean (SD)) & 58.0 (10.2) & 56.9 (9.9) & 56.5 (10.2) & 55.6 (10.1) & 55.6 (9.9) & 57.9 (9.9)\\
\rowcolor{gray!6}  \hspace{1em}Current smoker, \% & 6.9 & 4.2 & 6.1 & 5.5 & 4.3 & 3.5\\
\hspace{1em}Daily alcohol drinker, \% & 4.3 & 4.5 & 5.5 & 4.3 & 4.2 & 4.6\\
\rowcolor{gray!6}  \hspace{1em}BMI, kg/m$^2$ (mean (SD)) & 23.0 (3.4) & 22.9 (3.7) & 23.0 (3.8) & 23.1 (4.4) & 23.1 (3.1) & 22.8 (3.6)\\
\hspace{1em}Exercise (> 1h/week), \% & 13.6 & 20.8 & 17.1 & 18.5 & 18.8 & 22.6\\
\rowcolor{gray!6}  \hspace{1em}Sleep duration, 8-9 hours, \% & 27.7 & 25.6 & 25.1 & 25.9 & 25.4 & 25.7\\
\hspace{1em}Vegetable intake, daily, \% & 24.7 & 30.4 & 25.0 & 24.6 & 24.2 & 34.8\\
\rowcolor{gray!6}  \hspace{1em}Fruit intake, daily, \% & 25.0 & 35.7 & 26.6 & 29.2 & 29.2 & 41.1\\
\hspace{1em}Green tea intake, daily, \% & 73.8 & 76.8 & 77.0 & 76.4 & 75.8 & 77.3\\
\rowcolor{gray!6}  \hspace{1em}Coffee intake, daily, \% & 39.6 & 48.2 & 46.2 & 46.4 & 44.4 & 50.2\\
\hspace{1em}Educated over 18 years old, \% & 19.9 & 31.6 & 27.9 & 29.8 & 27.4 & 34.0\\
\rowcolor{gray!6}  \hspace{1em}History of diabetes, \% & 2.6 & 3.7 & 3.2 & 2.7 & 2.7 & 4.4\\
\hspace{1em}History of hypertension, \% & 21.5 & 19.7 & 20.5 & 19.1 & 18.9 & 20.0\\
\rowcolor{gray!6}  \hspace{1em}History of kidney diseases, \% & 3.6 & 4.1 & 3.9 & 3.7 & 3.7 & 4.4\\
\hspace{1em}History of liver diseases, \% & 3.5 & 4.6 & 4.9 & 3.9 & 3.9 & 5.0\\
\bottomrule
\end{tabular}}
\DIFaddendFL \end{table}

\hypertarget{results}{%
\section{Results}\label{results}}

The total follow-up was 1555073 person-years (median = 19.3 years),
during which 2675 death from stroke was confirmed (1352 men and 1323
women). Among these stroke mortality, 952 were hemorrhagic stroke (432
men and 520 women), and 957 were cerebral infarction (520 men and 437
women).

As listed in \textbf{Table \ref{tab:tab1f}}, compared with those who
chose ``never'' as their milk intake frequency at baseline, milk
drinkers were less likely to be a current smoker or a daily alcohol
consumer in both men and women. Furthermore, people consumed milk more
than 1-2 times/month were more likely to be a daily consumers of
vegetable, fruit as well as coffee, and more likely to join exercise
more than 1 hour/week among both sex.

Detailed results from the Bayesian survival models (crude, age-adjusted
and multivariable-adjusted) according to the frequency of milk intake
separated by sex are listed in \textbf{Table \ref{tab:tab2}} (men) and
\textbf{Table \ref{tab:tab3}} (women). Compared to those who never had
milk, both men and women had slower speed and lower hazard of dying from
total stroke in crude models. Velocities that milk consumers dying from
stroke is slower by a crude acceleration factor (AF) between 0.79 (sd =
0.05; 95\% CrI: 0.74, 0.90) and 0.93 (sd = 0.04; 95\% CrI: 0.85, 1.02)
compared with non-consumers. Chances that the posterior crude HRs were
estimated to be lower than 1 for those who had at least 1-2 times/month
was higher than 86.5\% in men and greater than 94.6\% in women. However,
lower hazard and delayed time-to-event was observed to remain after age
or multivariable adjustment only among daily male milk consumers.
Specifically, the mean (sd; 95\% CrI) of posterior
multivariable-adjusted AF and HR for daily male consumers of milk were
0.88 (sd = 0.05; 95\% CrI: 0.81, 0.96) and 0.80 (sd = 0.07; 95\% CrI:
0.69, 0.93) with a probability of 99.0\% to be smaller than the null
value (=1). Daily female milk consumers had posterior AFs and HRs that
was distributed with means of 0.97 (sd = 0.09; 95\% CrI: 0.88, 1.10) and
0.95 (sd = 0.12; 95\% CrI: 0.80, 1.17) which had about 78.0\% of chance
that their HRs could be smaller than 1.

Posterior distributions of AFs and HRs for mortality from hemorrhagic
stroke were found to contain the null value for either men or women
among all fitted models. In contrast, men who had milk intake frequency
higher than 1-2 times/week were found to be associated with averagely
17\%-20\% slower velocity or 28\%-39\% lower hazard of dying from
cerebral infarction compared to men who never drank milk (Model 2 in
\textbf{Table \ref{tab:tab2}}). Probability that the posterior HRs
distributed below the null value was greater or equal to 97.5\%. No
evidence was found about the associations between milk intake and hazard
of cerebral infarction mortality among women.

\hypertarget{discussion}{%
\section{Discussion}\label{discussion}}

In the JACC study cohort, our analyses showed that men in Japan who
consumed milk almost daily had lower hazard of dying from stroke
especially from cerebral infarction. Our evidence also suggested that
stroke mortality events were delayed among Japanese male daily milk
consumers compared with non-consumers.

\begin{table}[H]

\caption{\label{tab:tab2}Summary of posterior Acceleration Factors (AF) and Hazard Ratios (HR) of mortality from total stroke, stroke types according to the frequency of milk intake in men (JACC study, 1988-2009).}
\centering
\fontsize{6}{8}\selectfont
\resizebox*{!}{\dimexpr\textheight-\lineskip\relax}{%
\begin{tabular}[t]{lccccc}
\toprule
\textbf{ } & \textbf{Never} & \textbf{1-2 times/Month} & \textbf{1-2 times/Week} & \textbf{3-4 times/Week} & \textbf{Almost Daily}\\
\midrule
\rowcolor{gray!6}  Person-year & 135704 & 56551 & 97098 & 92153 & 252364\\
N & 8508 & 3522 & 5928 & 5563 & 15865\\
\rowcolor{gray!6}  Total Stroke & 326 & 122 & 181 & 177 & 546\\
\addlinespace[0.3em]
\multicolumn{6}{l}{\textbf{Model 0}}\\
\hspace{1em}Mean AF (SD) & 1 & 0.93 (0.07) & 0.83 (0.05) & 0.85 (0.05) & 0.93 (0.04)\\
\rowcolor{gray!6}  \hspace{1em}95\% CrI & - & (0.81, 1.06) & (0.73, 0.94) & (0.74, 0.96) & (0.85, 1.02)\\
\hspace{1em}Mean HR (SD) & 1 & 0.89 (0.09) & 0.77 (0.07) & 0.79 (0.07) & 0.90 (0.06)\\
\rowcolor{gray!6}  \hspace{1em}95\% CrI & - & (0.73, 1.08) & (0.63, 0.91) & (0.66, 0.94) & (0.79, 1.03)\\
\hspace{1em}Pr(HR < 1) & - & 86.5\% & 99.9\% & 99.7\% & 93.5\%\\
\addlinespace[0.3em]
\multicolumn{6}{l}{\textbf{Model 1}}\\
\rowcolor{gray!6}  \hspace{1em}Mean AF (SD) & 1 & 0.99 (0.06) & 0.90 (0.05) & 0.91 (0.05) & 0.85 (0.04)\\
\hspace{1em}95\% CrI & - & (0.87, 1.11) & (0.81, 1.00) & (0.82, 1.01) & (0.78, 0.92)\\
\rowcolor{gray!6}  \hspace{1em}Mean HR (SD) & 1 & 0.98 (0.11) & 0.84 (0.08) & 0.86 (0.08) & 0.76 (0.05)\\
\hspace{1em}95\% CrI & - & (0.79, 1.19) & (0.70, 1.00) & (0.71, 1.02) & (0.66, 0.87)\\
\rowcolor{gray!6}  \hspace{1em}Pr(HR < 1) & - & 58.7\% & 97.3\% & 96.1\% & 100.0\%\\
\addlinespace[0.3em]
\multicolumn{6}{l}{\textbf{Model 2}}\\
\hspace{1em}Mean AF (SD) & 1 & 1.00 (0.07) & 0.92 (0.06) & 0.94 (0.06) & 0.88 (0.05)\\
\rowcolor{gray!6}  \hspace{1em}95\% CrI & - & (0.88, 1.14) & (0.82, 1.03) & (0.84, 1.05) & (0.81, 0.96)\\
\hspace{1em}Mean HR (SD) & 1 & 1.01 (0.12) & 0.87 (0.09) & 0.90 (0.09) & 0.80 (0.07)\\
\rowcolor{gray!6}  \hspace{1em}95\% CrI & - & (0.81, 1.24) & (0.72, 1.05) & (0.74, 1.08) & (0.69, 0.93)\\
\hspace{1em}Pr(HR < 1) & - & 50.6\% & 93.7\% & 89.6\% & 99.0\%\\
\hline
\rowcolor{gray!6}  Hemorrhagic stroke & 100 & 42 & 58 & 56 & 176\\
\addlinespace[0.3em]
\multicolumn{6}{l}{\textbf{Model 0}}\\
\hspace{1em}Mean AF (SD) & 1 & 1.03 (0.17) & 0.85 (0.12) & 0.87 (0.13) & 0.98 (0.11)\\
\rowcolor{gray!6}  \hspace{1em}95\% CrI & - & (0.74, 1.38) & (0.63, 1.12) & (0.65, 1.14) & (0.78, 1.22)\\
\hspace{1em}Mean HR (SD) & 1 & 1.03 (0.19) & 0.82 (0.14) & 0.84 (0.15) & 0.97 (0.13)\\
\rowcolor{gray!6}  \hspace{1em}95\% CrI & - & (0.70, 1.46) & (0.56, 1.14) & (0.60, 1.17) & (0.75, 1.26)\\
\hspace{1em}Pr(HR < 1) & - & 47.2\% & 88.4\% & 86.3\% & 63.1\%\\
\addlinespace[0.3em]
\multicolumn{6}{l}{\textbf{Model 1}}\\
\rowcolor{gray!6}  \hspace{1em}Mean AF (SD) & 1 & 1.08 (0.17) & 0.91 (0.13) & 0.92 (0.13) & 0.90 (0.10)\\
\hspace{1em}95\% CrI & - & (0.80, 1.45) & (0.70, 1.20) & (0.71, 1.19) & (0.74, 1.11)\\
\rowcolor{gray!6}  \hspace{1em}Mean HR (SD) & 1 & 1.11 (0.21) & 0.88 (0.16) & 0.90 (0.16) & 0.88 (0.12)\\
\hspace{1em}95\% CrI & - & (0.75, 1.58) & (0.63, 1.25) & (0.63, 1.24) & (0.67, 1.14)\\
\rowcolor{gray!6}  \hspace{1em}Pr(HR < 1) & - & 31.6\% & 79.7\% & 76.6\% & 87.6\%\\
\addlinespace[0.3em]
\multicolumn{6}{l}{\textbf{Model 2}}\\
\hspace{1em}Mean AF (SD) & 1 & 1.11 (0.18) & 0.93 (0.15) & 0.96 (0.16) & 0.96 (0.13)\\
\rowcolor{gray!6}  \hspace{1em}95\% CrI & - & (0.79, 1.58) & (0.70, 1.25) & (0.71, 1.34) & (0.76, 1.25)\\
\hspace{1em}Mean HR (SD) & 1 & 1.14 (0.22) & 0.92 (0.17) & 0.95 (0.18) & 0.95 (0.14)\\
\rowcolor{gray!6}  \hspace{1em}95\% CrI & - & (0.75, 1.61) & (0.63, 1.29) & (0.65, 1.37) & (0.71, 1.27)\\
\hspace{1em}Pr(HR < 1) & - & 28.8\% & 72.4\% & 64.4\% & 69.3\%\\
\hline
\rowcolor{gray!6}  Cerebral infarction & 151 & 41 & 64 & 66 & 198\\
\addlinespace[0.3em]
\multicolumn{6}{l}{\textbf{Model 0}}\\
\hspace{1em}Mean AF (SD) & 1 & 0.76 (0.09) & 0.71 (0.07) & 0.74 (0.08) & 0.79 (0.06)\\
\rowcolor{gray!6}  \hspace{1em}95\% CrI & - & (0.59, 0.94) & (0.58, 0.86) & (0.61, 0.89) & (0.68, 0.93)\\
\hspace{1em}Mean HR (SD) & 1 & 0.65 (0.12) & 0.59 (0.09) & 0.64 (0.09) & 0.71 (0.09)\\
\rowcolor{gray!6}  \hspace{1em}95\% CrI & - & (0.46, 0.92) & (0.43, 0.79) & (0.47, 0.85) & (0.56, 0.89)\\
\hspace{1em}Pr(HR < 1) & - & 99.1\% & 99.9\% & 99.7\% & 99.5\%\\
\addlinespace[0.3em]
\multicolumn{6}{l}{\textbf{Model 1}}\\
\rowcolor{gray!6}  \hspace{1em}Mean AF (SD) & 1 & 0.83 (0.08) & 0.79 (0.07) & 0.82 (0.07) & 0.74 (0.05)\\
\hspace{1em}95\% CrI & - & (0.68, 1.01) & (0.67, 0.93) & (0.69, 0.96) & (0.66, 0.84)\\
\rowcolor{gray!6}  \hspace{1em}Mean HR (SD) & 1 & 0.73 (0.13) & 0.65 (0.10) & 0.70 (0.11) & 0.58 (0.07)\\
\hspace{1em}95\% CrI & - & (0.49, 1.02) & (0.48, 0.88) & (0.51, 0.94) & (0.46, 0.72)\\
\rowcolor{gray!6}  \hspace{1em}Pr(HR < 1) & - & 96.9\% & 99.8\% & 98.9\% & 100.0\%\\
\addlinespace[0.3em]
\multicolumn{6}{l}{\textbf{Model 2}}\\
\hspace{1em}Mean AF (SD) & 1 & 0.84 (0.09) & 0.80 (0.08) & 0.83 (0.08) & 0.75 (0.06)\\
\rowcolor{gray!6}  \hspace{1em}95\% CrI & - & (0.67, 1.02) & (0.67, 0.95) & (0.69, 0.99) & (0.66, 0.85)\\
\hspace{1em}Mean HR (SD) & 1 & 0.73 (0.14) & 0.67 (0.11) & 0.72 (0.12) & 0.61 (0.08)\\
\rowcolor{gray!6}  \hspace{1em}95\% CrI & - & (0.50, 1.04) & (0.48, 0.91) & (0.52, 0.99) & (0.48, 0.79)\\
\hspace{1em}Pr(HR < 1) & - & 96.1\% & 99.1\% & 97.5\% & 99.8\%\\
\bottomrule
\multicolumn{6}{l}{\textit{Note: }}\\
\multicolumn{6}{l}{Abbreviations: SD, standard deviation; CrI, credible interval; MCSE, Monte Carlo Standard Error;}\\
\multicolumn{6}{l}{ Pr(HR < 1) indicates the prabability for posterior HR to be smaller than 1.}\\
\multicolumn{6}{l}{Model 0 = Crude model; Model 1 = age-adjusted model; Model 2 = multivariable adjusted model.}\\
\multicolumn{6}{l}{Covariates included in Model 2: age, smoking habit, alcohol intake, body mass index, history of}\\
\multicolumn{6}{l}{hypertension, diabetes, kidney/liver diseases, exercise, sleep duration, coffee intake, education level.}\\
\end{tabular}}
\end{table}

\begin{table}[H]

\caption{\label{tab:tab3}Summary of posterior Acceleration Factors (AF) and Hazard Ratios (HR) of mortality from total stroke, stroke type according to the frequency of milk intake in women (JACC study, 1988-2009).}
\centering
\fontsize{7}{9}\selectfont
\resizebox*{!}{\dimexpr\textheight-\lineskip\relax}{%
\begin{tabular}[t]{lccccc}
\toprule
\textbf{ } & \textbf{Never} & \textbf{1-2 times/Month} & \textbf{1-2 times/Week} & \textbf{3-4 times/Week} & \textbf{Almost Daily}\\
\midrule
\rowcolor{gray!6}  Person-year & 173222 & 59904 & 129233 & 139919 & 418925\\
N & 10407 & 3640 & 7590 & 8108 & 25254\\
\rowcolor{gray!6}  Total Stroke & 300 & 84 & 182 & 172 & 585\\
\addlinespace[0.3em]
\multicolumn{6}{l}{\textbf{Model 0}}\\
\hspace{1em}Mean AF (SD) & 1 & 0.88 (0.07) & 0.87 (0.05) & 0.79 (0.05) & 0.88 (0.04)\\
\rowcolor{gray!6}  \hspace{1em}95\% CrI & - & (0.75, 1.03) & (0.78, 0.98) & (0.71, 0.90) & (0.80, 0.96)\\
\hspace{1em}Mean HR (SD) & 1 & 0.83 (0.10) & 0.81 (0.08) & 0.70 (0.07) & 0.81 (0.07)\\
\rowcolor{gray!6}  \hspace{1em}95\% CrI & - & (0.64, 1.05) & (0.68, 0.97) & (0.58, 0.85) & (0.71, 0.93)\\
\hspace{1em}Pr(HR < 1) & - & 94.6\% & 98.7\% & 99.9\% & 99.6\%\\
\addlinespace[0.3em]
\multicolumn{6}{l}{\textbf{Model 1}}\\
\rowcolor{gray!6}  \hspace{1em}Mean AF (SD) & 1 & 0.99 (0.09) & 1.11 (0.08) & 1.02 (0.08) & 0.95 (0.06)\\
\hspace{1em}95\% CrI & - & (0.85, 1.17) & (0.97, 1.26) & (0.89, 1.16) & (0.86, 1.06)\\
\rowcolor{gray!6}  \hspace{1em}Mean HR (SD) & 1 & 1.00 (0.14) & 1.18 (0.14) & 1.03 (0.12) & 0.92 (0.09)\\
\hspace{1em}95\% CrI & - & (0.76, 1.31) & (0.95, 1.47) & (0.82, 1.28) & (0.78, 1.09)\\
\rowcolor{gray!6}  \hspace{1em}Pr(HR < 1) & - & 52.3\% & 6.3\% & 42.0\% & 86.8\%\\
\addlinespace[0.3em]
\multicolumn{6}{l}{\textbf{Model 2}}\\
\hspace{1em}Mean AF (SD) & 1 & 1.01 (0.12) & 1.11 (0.14) & 1.02 (0.12) & 0.97 (0.09)\\
\rowcolor{gray!6}  \hspace{1em}95\% CrI & - & (0.85, 1.20) & (0.97, 1.30) & (0.89, 1.19) & (0.88, 1.10)\\
\hspace{1em}Mean HR (SD) & 1 & 1.01 (0.17) & 1.19 (0.15) & 1.03 (0.15) & 0.95 (0.12)\\
\rowcolor{gray!6}  \hspace{1em}95\% CrI & - & (0.75, 1.36) & (0.96, 1.52) & (0.81, 1.31) & (0.80, 1.17)\\
\hspace{1em}Pr(HR < 1) & - & 52.8\% & 6.4\% & 44.4\% & 78.0\%\\
\hline
\rowcolor{gray!6}  Hemorrhagic stroke & 108 & 27 & 78 & 76 & 231\\
\addlinespace[0.3em]
\multicolumn{6}{l}{\textbf{Model 0}}\\
\hspace{1em}Mean AF (SD) & 1 & 0.78 (0.13) & 0.98 (0.12) & 0.90 (0.11) & 0.92 (0.09)\\
\rowcolor{gray!6}  \hspace{1em}95\% CrI & - & (0.55, 1.06) & (0.76, 1.25) & (0.70, 1.13) & (0.76, 1.12)\\
\hspace{1em}Mean HR (SD) & 1 & 0.73 (0.16) & 0.98 (0.15) & 0.87 (0.14) & 0.89 (0.11)\\
\rowcolor{gray!6}  \hspace{1em}95\% CrI & - & (0.47, 1.08) & (0.71, 1.31) & (0.64, 1.16) & (0.71, 1.15)\\
\hspace{1em}Pr(HR < 1) & - & 94.7\% & 58.1\% & 83.1\% & 83.0\%\\
\addlinespace[0.3em]
\multicolumn{6}{l}{\textbf{Model 1}}\\
\rowcolor{gray!6}  \hspace{1em}Mean AF (SD) & 1 & 0.88 (0.13) & 1.12 (0.13) & 1.04 (0.13) & 0.95 (0.09)\\
\hspace{1em}95\% CrI & - & (0.63, 1.17) & (0.90, 1.41) & (0.82, 1.32) & (0.80, 1.14)\\
\rowcolor{gray!6}  \hspace{1em}Mean HR (SD) & 1 & 0.84 (0.18) & 1.17 (0.18) & 1.06 (0.17) & 0.93 (0.12)\\
\hspace{1em}95\% CrI & - & (0.54, 1.24) & (0.86, 1.58) & (0.76, 1.45) & (0.73, 1.19)\\
\rowcolor{gray!6}  \hspace{1em}Pr(HR < 1) & - & 81.6\% & 16.9\% & 38.9\% & 74.6\%\\
\addlinespace[0.3em]
\multicolumn{6}{l}{\textbf{Model 2}}\\
\hspace{1em}Mean AF (SD) & 1 & 0.93 (0.24) & 1.23 (0.38) & 1.14 (0.33) & 1.04 (0.25)\\
\rowcolor{gray!6}  \hspace{1em}95\% CrI & - & (0.64, 1.33) & (0.93, 1.98) & (0.87, 1.83) & (0.83, 1.55)\\
\hspace{1em}Mean HR (SD) & 1 & 0.89 (0.22) & 1.26 (0.26) & 1.15 (0.23) & 1.02 (0.19)\\
\rowcolor{gray!6}  \hspace{1em}95\% CrI & - & (0.55, 1.39) & (0.90, 1.90) & (0.83, 1.74) & (0.78, 1.51)\\
\hspace{1em}Pr(HR < 1) & - & 73.2\% & 9.5\% & 24.8\% & 53.3\%\\
\hline
\rowcolor{gray!6}  Cerebral infarction & 102 & 35 & 63 & 50 & 187\\
\addlinespace[0.3em]
\multicolumn{6}{l}{\textbf{Model 0}}\\
\hspace{1em}Mean AF (SD) & 1 & 1.01 (0.13) & 0.90 (0.09) & 0.75 (0.08) & 0.86 (0.06)\\
\rowcolor{gray!6}  \hspace{1em}95\% CrI & - & (0.79, 1.27) & (0.75, 1.10) & (0.60, 0.91) & (0.75, 0.99)\\
\hspace{1em}Mean HR (SD) & 1 & 1.03 (0.20) & 0.85 (0.14) & 0.61 (0.11) & 0.78 (0.10)\\
\rowcolor{gray!6}  \hspace{1em}95\% CrI & - & (0.69, 1.48) & (0.60, 1.13) & (0.43, 0.84) & (0.59, 0.99)\\
\hspace{1em}Pr(HR < 1) & - & 51.9\% & 75.6\% & 97.6\% & 96.1\%\\
\addlinespace[0.3em]
\multicolumn{6}{l}{\textbf{Model 1}}\\
\rowcolor{gray!6}  \hspace{1em}Mean AF (SD) & 1 & 1.21 (0.32) & 1.16 (0.30) & 0.98 (0.19) & 0.97 (0.14)\\
\hspace{1em}95\% CrI & - & (0.95, 2.08) & (0.93, 1.95) & (0.79, 1.48) & (0.84, 1.43)\\
\rowcolor{gray!6}  \hspace{1em}Mean HR (SD) & 1 & 1.37 (0.33) & 1.25 (0.28) & 0.94 (0.22) & 0.92 (0.17)\\
\hspace{1em}95\% CrI & - & (0.89, 2.18) & (0.87, 1.95) & (0.63, 1.52) & (0.69, 1.40)\\
\rowcolor{gray!6}  \hspace{1em}Pr(HR < 1) & - & 8.5\% & 14.2\% & 70.1\% & 79.4\%\\
\addlinespace[0.3em]
\multicolumn{6}{l}{\textbf{Model 2}}\\
\hspace{1em}Mean AF (SD) & 1 & 1.19 (0.19) & 1.12 (0.15) & 0.96 (0.12) & 0.97 (0.09)\\
\rowcolor{gray!6}  \hspace{1em}95\% CrI & - & (0.94, 1.62) & (0.92, 1.49) & (0.78, 1.21) & (0.83, 1.18)\\
\hspace{1em}Mean HR (SD) & 1 & 1.38 (0.29) & 1.21 (0.22) & 0.91 (0.18) & 0.94 (0.14)\\
\rowcolor{gray!6}  \hspace{1em}95\% CrI & - & (0.89, 2.02) & (0.85, 1.70) & (0.62, 1.34) & (0.69, 1.25)\\
\hspace{1em}Pr(HR < 1) & - & 7.3\% & 15.6\% & 72.8\% & 70.0\%\\
\bottomrule
\multicolumn{6}{l}{\textit{Note: }}\\
\multicolumn{6}{l}{Abbreviations: SD, standard deviation; CrI, credible interval; MCSE, Monte Carlo Standard Error;}\\
\multicolumn{6}{l}{ Pr(HR < 1) indicates the prabability for posterior HR to be smaller than 1.}\\
\multicolumn{6}{l}{Model 0 = Crude model; Model 1 = age-adjusted model; Model 2 = multivariable adjusted model.}\\
\multicolumn{6}{l}{Covariates included in Model 2: age, smoking habit, alcohol intake, body mass index, history of}\\
\multicolumn{6}{l}{hypertension, diabetes, kidney/liver diseases, exercise, sleep duration, coffee intake, education level.}\\
\end{tabular}}
\end{table}

These findings are in line with our previous report
\citep{wang_milk_2015} as well as other studies conducted in East Asian
populations
\citep{umesawa2008dietary, kondo2013consumption, ozawa2017dietary, sauvaget2003intake, Talaei_2016}.
Moreover, we have further updated with more comprehensive and
straightforward evidence about whether and how certain the data had
shown about daily consumption of milk is contributing to a postponed
stroke (mostly cerebral infarction) mortality event among Japanese men.
A recent dose-response meta-analysis of 18 prospective cohort studies
had also shown a similar negative association \citep{DeGoede2016}
between milk consumption and risk of stroke. The same meta-analysis also
reported a greater reduction of risk of stroke (18\%) for East Asian
population in contrast with the 7\% less risk in the pooled overall
finding for all populations combined. Benefits of increased milk intake
might be particularly noticeable in East Asian countries where strokes
are relatively more common, and milk consumption is much lower than
those studies conducted among European or American populations
\citep{dehghan2018association}.

Possible reasons for a protective effect of milk consumption against
stroke could be interpreted as such an association may be mediated by
its content in calcium, magnesium, potassium, and other bioactive
compounds, as recommended by \citet{Iacoviello2018}. Apart from the
inorganic minerals in milk that would be helpful with health effects,
recent studies on animal models also indicated key evidence that
stroke-associated morbidity was delayed in stroke-prone rats who were
fed with milk-protein enriched diets \citep{Chiba2012, singh2016diets}.
More precisely, \citet{Singh2020} found that whey protein and its
components lactalbumin and lactoferrin improved energy balance and
glycemic control against the onset of neurological deficits associated
with stroke. Bioactive peptides from milk proteins were also responsible
for limitation of thrombosis \citep{tokajuk2019whey} through their
angiotensin convertase enzyme inhibitory potential, which might partly
explain why the effect was found mainly for mortality from cerebral
infarction in the current study.

Some limitations here are worth mentioning. First, although our object
was not to answer which type of milk is protective, but if such
information were somehow available in the JACC study database, more
detailed comparison or stratification would have been possible. Second,
despite reasonable validity of FFQ in the JACC study cohort was assessed
and confirmed, measurement errors are inevitable. Therefore, we did not
try to compute the amount of consumption by multiplying an average
volume per occasion with the frequency of intake since the random error
might be exaggerated and the observed associations may have attenuated.
Strengths of our analyses included that we have transformed the research
questions to more transparent ones that is easier for interpretation.
Direct probabilities that daily milk intake is associated with lower
hazard or delayed stroke mortality event were provided here after
thorough computer simulation.

In conclusion, the JACC study database has provided evidence that
Japanese men who consumed milk daily had lower hazard of dying from
stroke especially cerebral infarction compared with their counterparts
who never consumed milk. Time before an event of stroke mortality
occurred were slowed down and delayed among men who drank milk
regularly.

% %%%%%%%%%%%%%%%%%%%%%%%%%%%%%%%%%%%%%%%%%%
% %% optional
% \supplementary{The following are available online at www.mdpi.com/link, Figure S1: title, Table S1: title, Video S1: title.}
%
% % Only for the journal Methods and Protocols:
% % If you wish to submit a video article, please do so with any other supplementary material.
% % \supplementary{The following are available at www.mdpi.com/link: Figure S1: title, Table S1: title, Video S1: title. A supporting video article is available at doi: link.}

\vspace{6pt}

%%%%%%%%%%%%%%%%%%%%%%%%%%%%%%%%%%%%%%%%%%
\acknowledgments{The authors would like to express their sincere
appreciate to Kunio Aoki and Yoshiyuki Ohno, Professors Emeritus at
Nagoya University School of Medicine and former chairpersons of the JACC
Study. The whole member of JACC Study Group can be found at
\url{https://publichealth.med.hokudai.ac.jp/jacc/index.html}. The JACC
Study has been supported by Grants-in-Aid for Scientific Research from
the Ministry of Education, Culture, Sports, Science and Technology of
Japan (MEXT, Monbu Kagaku-sho), Tokyo (grant numbers 61010076, 62010074,
63010074, 1010068, 2151065, 3151064, 4151063, 5151069, 6279102,
11181101, 17015022, 18014011, 20014026, 20390156, 26293138, and
16H06277), and Grants-in-Aid from the Ministry of Health, Labour and
Welfare, Health and Labour Sciences Research Grants, Japan
{[}H20-Junkankitou (Seishuu)-Ippan-013, H23-Junkankitou (Seishuu)-
Ippan-005, H26-Junkankitou (Seisaku)-Ippan-001, and H29-Junkankitou
(Seishuu)-Ippan-003{]}.}

%%%%%%%%%%%%%%%%%%%%%%%%%%%%%%%%%%%%%%%%%%
\authorcontributions{``C.W. and H.Y. conceived and designed the study;
C.W. analyzed the data; C.W. wrote the first draft of the paper. A.T.
provided the database. All of the authors approved and finalized the
manuscript for publication.'\,'}

%%%%%%%%%%%%%%%%%%%%%%%%%%%%%%%%%%%%%%%%%%
\conflictsofinterest{The authors declare no conflict of interest. The
founding sponsors had no role in the design of the study; in the
collection, analyses, or interpretation of data; in the writing of the
manuscript, an in the decision to publish the results.}

%%%%%%%%%%%%%%%%%%%%%%%%%%%%%%%%%%%%%%%%%%
%% optional
\abbreviations{The following abbreviations are used in this manuscript:\\

\noindent
\begin{tabular}{@{}ll}
JACC & Japan Collaborative Cohort \\
FFQ & Food Frequency Questionnaire \\
MCMC & Markov Chain Monte Carlo \\
JAGS & Just Another Gibbs Samplers \\
AFT & accelerated failure time \\
HR & hazard ratio \\
AF & acceleration factor \\
sd & standard deviation \\
CrI & credible interval \\
\end{tabular}}


%%%%%%%%%%%%%%%%%%%%%%%%%%%%%%%%%%%%%%%%%%
% Citations and References in Supplementary files are permitted provided that they also appear in the reference list here.

%=====================================
% References, variant A: internal bibliography
%=====================================
%\reftitle{References}
%\begin{thebibliography}{999}
% Reference 1
%\bibitem[Author1(year)]{ref-journal}
%Author1, T. The title of the cited article. {\em Journal Abbreviation} {\bf 2008}, {\em 10}, 142--149.
% Reference 2
%\bibitem[Author2(year)]{ref-book}
%Author2, L. The title of the cited contribution. In {\em The Book Title}; Editor1, F., Editor2, A., Eds.; Publishing House: City, Country, 2007; pp. 32--58.
%\end{thebibliography}

% The following MDPI journals use author-date citation: Arts, Econometrics, Economies, Genealogy, Humanities, IJFS, JRFM, Laws, Religions, Risks, Social Sciences. For those journals, please follow the formatting guidelines on http://www.mdpi.com/authors/references
% To cite two works by the same author: \citeauthor{ref-journal-1a} (\citeyear{ref-journal-1a}, \citeyear{ref-journal-1b}). This produces: Whittaker (1967, 1975)
% To cite two works by the same author with specific pages: \citeauthor{ref-journal-3a} (\citeyear{ref-journal-3a}, p. 328; \citeyear{ref-journal-3b}, p.475). This produces: Wong (1999, p. 328; 2000, p. 475)

%=====================================
% References, variant B: external bibliography
%=====================================
\reftitle{References}
\externalbibliography{yes}
\bibliography{milk.bib}

%%%%%%%%%%%%%%%%%%%%%%%%%%%%%%%%%%%%%%%%%%
%% optional

%% for journal Sci
%\reviewreports{\\
%Reviewer 1 comments and authors’ response\\
%Reviewer 2 comments and authors’ response\\
%Reviewer 3 comments and authors’ response
%}

%%%%%%%%%%%%%%%%%%%%%%%%%%%%%%%%%%%%%%%%%%
\end{document}
