%  LaTeX support: latex@mdpi.com
%  In case you need support, please attach all files that are necessary for compiling as well as the log file, and specify the details of your LaTeX setup (which operating system and LaTeX version / tools you are using).

%=================================================================
\documentclass[nutrients,article,submit,moreauthors,pdftex]{mdpi}

% If you would like to post an early version of this manuscript as a preprint, you may use preprint as the journal and change 'submit' to 'accept'. The document class line would be, e.g., \documentclass[preprints,article,accept,moreauthors,pdftex]{mdpi}. This is especially recommended for submission to arXiv, where line numbers should be removed before posting. For preprints.org, the editorial staff will make this change immediately prior to posting.

%% Some pieces required from the pandoc template
\providecommand{\tightlist}{%
  \setlength{\itemsep}{0pt}\setlength{\parskip}{4pt}}
\setlist[itemize]{leftmargin=*,labelsep=5.8mm}
\setlist[enumerate]{leftmargin=*,labelsep=4.9mm}

\usepackage{longtable}

% see https://stackoverflow.com/a/47122900

%--------------------
% Class Options:
%--------------------
%----------
% journal
%----------
% Choose between the following MDPI journals:
% acoustics, actuators, addictions, admsci, aerospace, agriculture, agriengineering, agronomy, algorithms, animals, antibiotics, antibodies, antioxidants, applsci, arts, asc, asi, atmosphere, atoms, axioms, batteries, bdcc, behavsci , beverages, bioengineering, biology, biomedicines, biomimetics, biomolecules, biosensors, brainsci , buildings, cancers, carbon , catalysts, cells, ceramics, challenges, chemengineering, chemistry, chemosensors, children, cleantechnol, climate, clockssleep, cmd, coatings, colloids, computation, computers, condensedmatter, cosmetics, cryptography, crystals, dairy, data, dentistry, designs , diagnostics, diseases, diversity, drones, econometrics, economies, education, electrochem, electronics, energies, entropy, environments, epigenomes, est, fermentation, fibers, fire, fishes, fluids, foods, forecasting, forests, fractalfract, futureinternet, futurephys, galaxies, games, gastrointestdisord, gels, genealogy, genes, geohazards, geosciences, geriatrics, hazardousmatters, healthcare, heritage, highthroughput, horticulturae, humanities, hydrology, ijerph, ijfs, ijgi, ijms, ijns, ijtpp, informatics, information, infrastructures, inorganics, insects, instruments, inventions, iot, j, jcdd, jcm, jcp, jcs, jdb, jfb, jfmk, jimaging, jintelligence, jlpea, jmmp, jmse, jnt, jof, joitmc, jpm, jrfm, jsan, land, languages, laws, life, literature, logistics, lubricants, machines, magnetochemistry, make, marinedrugs, materials, mathematics, mca, medicina, medicines, medsci, membranes, metabolites, metals, microarrays, micromachines, microorganisms, minerals, modelling, molbank, molecules, mps, mti, nanomaterials, ncrna, neuroglia, nitrogen, notspecified, nutrients, ohbm, particles, pathogens, pharmaceuticals, pharmaceutics, pharmacy, philosophies, photonics, physics, plants, plasma, polymers, polysaccharides, preprints , proceedings, processes, proteomes, psych, publications, quantumrep, quaternary, qubs, reactions, recycling, religions, remotesensing, reports, resources, risks, robotics, safety, sci, scipharm, sensors, separations, sexes, signals, sinusitis, smartcities, sna, societies, socsci, soilsystems, sports, standards, stats, surfaces, surgeries, sustainability, symmetry, systems, technologies, test, toxics, toxins, tropicalmed, universe, urbansci, vaccines, vehicles, vetsci, vibration, viruses, vision, water, wem, wevj

%---------
% article
%---------
% The default type of manuscript is "article", but can be replaced by:
% abstract, addendum, article, benchmark, book, bookreview, briefreport, casereport, changes, comment, commentary, communication, conceptpaper, conferenceproceedings, correction, conferencereport, expressionofconcern, extendedabstract, meetingreport, creative, datadescriptor, discussion, editorial, essay, erratum, hypothesis, interestingimages, letter, meetingreport, newbookreceived, obituary, opinion, projectreport, reply, retraction, review, perspective, protocol, shortnote, supfile, technicalnote, viewpoint
% supfile = supplementary materials

%----------
% submit
%----------
% The class option "submit" will be changed to "accept" by the Editorial Office when the paper is accepted. This will only make changes to the frontpage (e.g., the logo of the journal will get visible), the headings, and the copyright information. Also, line numbering will be removed. Journal info and pagination for accepted papers will also be assigned by the Editorial Office.

%------------------
% moreauthors
%------------------
% If there is only one author the class option oneauthor should be used. Otherwise use the class option moreauthors.

%---------
% pdftex
%---------
% The option pdftex is for use with pdfLaTeX. If eps figures are used, remove the option pdftex and use LaTeX and dvi2pdf.

%=================================================================
\firstpage{1}
\makeatletter
\setcounter{page}{\@firstpage}
\makeatother
\pubvolume{xx}
\issuenum{1}
\articlenumber{5}
\pubyear{2019}
\copyrightyear{2019}
%\externaleditor{Academic Editor: name}
\history{Received: date; Accepted: date; Published: date}
\updates{yes} % If there is an update available, un-comment this line

%% MDPI internal command: uncomment if new journal that already uses continuous page numbers
%\continuouspages{yes}

%------------------------------------------------------------------
% The following line should be uncommented if the LaTeX file is uploaded to arXiv.org
%\pdfoutput=1

%=================================================================
% Add packages and commands here. The following packages are loaded in our class file: fontenc, calc, indentfirst, fancyhdr, graphicx, lastpage, ifthen, lineno, float, amsmath, setspace, enumitem, mathpazo, booktabs, titlesec, etoolbox, amsthm, hyphenat, natbib, hyperref, footmisc, geometry, caption, url, mdframed, tabto, soul, multirow, microtype, tikz

%=================================================================
%% Please use the following mathematics environments: Theorem, Lemma, Corollary, Proposition, Characterization, Property, Problem, Example, ExamplesandDefinitions, Hypothesis, Remark, Definition
%% For proofs, please use the proof environment (the amsthm package is loaded by the MDPI class).

%=================================================================
% Full title of the paper (Capitalized)
\Title{Milk intake and risk of mortality risk in the Japan Collaborative
Cohort Study - a Bayesian survival analysis}

% Authors, for the paper (add full first names)
\Author{Chaochen
Wang$^{1,*}$\href{https://orcid.org/0000-0001-5533-1497}{\orcidicon}, Hiroshi
Yatsuya$^{2}$\href{https://orcid.org/0000-0002-6220-9251}{\orcidicon}, Yingsong
Lin$^{1}$\href{https://orcid.org/0000-0003-0214-3649}{\orcidicon}, Tae
Sasakabe$^{1}$\href{https://orcid.org/0000-0002-7417-5693}{\orcidicon}, Sayo
Kawai$^{1}$\href{https://orcid.org/0000-0002-5156-4880}{\orcidicon}, Shogo
Kikuchi$^{1}$\href{https://orcid.org/0000-0001-9499-9668}{\orcidicon}, Hiroyasu
Iso$^{3}$\href{https://orcid.org/0000-0002-9241-7289}{\orcidicon}, Akiko
Tamakoshi$^{4}$\href{https://orcid.org/0000-0002-9761-3879}{\orcidicon}}

% Authors, for metadata in PDF
\AuthorNames{Chaochen Wang, Hiroshi Yatsuya, Yingsong Lin, Tae
Sasakabe, Sayo Kawai, Shogo Kikuchi, Hiroyasu Iso, Akiko Tamakoshi}

% Affiliations / Addresses (Add [1] after \address if there is only one affiliation.)
\address{%
$^{1}$ \quad Department of Public Health, Aichi Medical University
School of Medicine, Nagakute, Japan; \\
$^{2}$ \quad Departmet of Public Health, Fujita Health University School
of Medicine, Toyoake, Japan; \\
$^{3}$ \quad Public Health, Department of Social Medicine, Osaka
University Graduate School of Medicine, Osaka, Japan; \\
$^{4}$ \quad Department of Public Health, Faculty of Medicine, Hokkaido
University, Sapporo, Japan; \\
}
% Contact information of the corresponding author
\corres{Correspondence: Email.:
\href{mailto:chaochen@wangcc.me}{\nolinkurl{chaochen@wangcc.me}}; Tel.:
+81-561-62-3311. Department of Public Health, Aichi Medical University
School of Medicine, 1-1 Yazakokarimata, Nagakute, Aichi, 480-1195, Japan
(C.W.)}

% Current address and/or shared authorship








% The commands \thirdnote{} till \eighthnote{} are available for further notes

% Simple summary

% Abstract (Do not insert blank lines, i.e. \\)
\abstract{A single paragraph of about 200 words maximum. For research
articles, abstracts should give a pertinent overview of the work. We
strongly encourage authors to use the following style of structured
abstracts, but without headings: 1) Background: Place the question
addressed in a broad context and highlight the purpose of the study; 2)
Methods: Describe briefly the main methods or treatments applied; 3)
Results: Summarize the article's main findings; and 4) Conclusion:
Indicate the main conclusions or interpretations. The abstract should be
an objective representation of the article, it must not contain results
which are not presented and substantiated in the main text and should
not exaggerate the main conclusions.}

% Keywords
\keyword{keyword 1; keyword 2; keyword 3 (list three to ten pertinent
keywords specific to the article, yet reasonably common within the
subject discipline.).}

% The fields PACS, MSC, and JEL may be left empty or commented out if not applicable
%\PACS{J0101}
%\MSC{}
%\JEL{}

%%%%%%%%%%%%%%%%%%%%%%%%%%%%%%%%%%%%%%%%%%
% Only for the journal Diversity
%\LSID{\url{http://}}

%%%%%%%%%%%%%%%%%%%%%%%%%%%%%%%%%%%%%%%%%%
% Only for the journal Applied Sciences:
%\featuredapplication{Authors are encouraged to provide a concise description of the specific application or a potential application of the work. This section is not mandatory.}
%%%%%%%%%%%%%%%%%%%%%%%%%%%%%%%%%%%%%%%%%%

%%%%%%%%%%%%%%%%%%%%%%%%%%%%%%%%%%%%%%%%%%
% Only for the journal Data:
%\dataset{DOI number or link to the deposited data set in cases where the data set is published or set to be published separately. If the data set is submitted and will be published as a supplement to this paper in the journal Data, this field will be filled by the editors of the journal. In this case, please make sure to submit the data set as a supplement when entering your manuscript into our manuscript editorial system.}

%\datasetlicense{license under which the data set is made available (CC0, CC-BY, CC-BY-SA, CC-BY-NC, etc.)}

%%%%%%%%%%%%%%%%%%%%%%%%%%%%%%%%%%%%%%%%%%
% Only for the journal Toxins
%\keycontribution{The breakthroughs or highlights of the manuscript. Authors can write one or two sentences to describe the most important part of the paper.}

%\setcounter{secnumdepth}{4}
%%%%%%%%%%%%%%%%%%%%%%%%%%%%%%%%%%%%%%%%%%


\usepackage{booktabs}
\usepackage{longtable}
\usepackage{array}
\usepackage{multirow}
\usepackage{wrapfig}
\usepackage{float}
\usepackage{colortbl}
\usepackage{pdflscape}
\usepackage{tabu}
\usepackage{threeparttable}
\usepackage{threeparttablex}
\usepackage[normalem]{ulem}
\usepackage{makecell}
\usepackage{xcolor}

\begin{document}
%%%%%%%%%%%%%%%%%%%%%%%%%%%%%%%%%%%%%%%%%%

\hypertarget{introduction}{%
\section{Introduction}\label{introduction}}

\hypertarget{materials-and-methods}{%
\section{Materials and Methods}\label{materials-and-methods}}

\hypertarget{the-database}{%
\subsection{The database}\label{the-database}}

We used data from the Japan Collaborative Cohort (JACC) study, which was
sponsored by the Ministry of Education, Sports, Science, and Technology
of Japan. Sampling methods and details about the JACC study have been
described extensively in the literature
\citep{Ohno2001, Tamakoshi2005, Tamakoshi2013}. Participants of the JACC
study completed self-administered questionnaires about their lifestyles,
food intake (food frequency questionnaire, FFQ), and medical histories
of cardiovascular disease or cancer. In the final follow-up of the JACC
study, data from a total of 110585 individuals (46395 men and 64190
women) were successfully retained for the current analysis. We further
excluded samples if they meet one of the following criteria: 1) with any
disease history of stroke, cancer, myocardial infarction, ischemic heart
disease, or other types heart disease (n = 6655, 2931 men and 3724
women); 2) did not answer the question regarding their milk consumption
in the baseline FFQ survey (n = 9545, 3593 men and 5952 women). Finally,
94385 (39386 men and 54999 women) are left in the database. The study
design and informed consent procedure were approved by the Ethics Review
Committee of Nagoya University School of Medicine.

\hypertarget{exposure-and-the-outcome-of-interest}{%
\subsection{Exposure and the outcome of
interest}\label{exposure-and-the-outcome-of-interest}}

Frequency of milk intake during the preceding year of the baseline was
assessed by FFQ from ``never'', ``1-2 times/month'', ``1-2 times/week'',
``3-4 times/week'', and ``Almost daily''. The exact amount of milk
consumption was difficult to assess here. However, good reproducibility
and validity were confirmed previously (Spearman rank correlation
coefficient between milk intake frequency and weighed dietary record for
12 days was 0.65) \citep{Date2005}.

The causes and date of death were obtained from death certificates and
were systematically reviewed. The follow-up period was defined as from
the time of the baseline survey was completed, which was between
1988-1990, until the end of 2009 (administrative censor), or the date
when move-out of study area, or the date of death from stroke recorded,
whichever occurred first. Other causes of death were treated as censored
and assumed not informative. The causes of death were coded by the 10th
Revision of the International Statistical Classification of Diseases and
Related Health Problems (ICD-10), therefore stroke was defined as
I60-I69. We further classified these deaths into hemorrhagic stroke
(I60, I61 and I62) or cerebral infarction (I63) when subtypes of stroke
in their death certificate were available.

\hypertarget{statistical-approach}{%
\subsection{Statistical approach}\label{statistical-approach}}

\hypertarget{results}{%
\section{Results}\label{results}}

\hypertarget{subsection-heading-here}{%
\subsection{Subsection Heading Here}\label{subsection-heading-here}}

\hypertarget{subsubsection-heading-here}{%
\subsubsection{Subsubsection Heading
Here}\label{subsubsection-heading-here}}

\begin{table}[!h]

\caption{\label{tab:tab1}}
\centering
\fontsize{8}{10}\selectfont
\begin{tabular}[t]{llllllllll}
\toprule
\multicolumn{1}{c}{ } & \multicolumn{5}{c}{Hazard ratio (HR)} & \multicolumn{4}{c}{Acceleration factor (AF)} \\
\cmidrule(l{3pt}r{3pt}){2-6} \cmidrule(l{3pt}r{3pt}){7-10}
Milk intake & Median & Mean (SD) & 95\% CrI & MCSE & Probability & Median & Mean (SD) & 95\% CrI & MCSE\\
\midrule
\rowcolor{gray!6}  Never & - & - & - & - & - & - & - & - & \vphantom{2} -\\
1-2 t/Mon & 0.88 & 0.89 (0.09) & (0.73, 1.08) & 0.0022 & 86.50\% & 0.93 & 0.93 (0.06) & (0.81, 1.06) & 0.0016\\
\rowcolor{gray!6}  1-2 t/Week & 0.77 & 0.77 (0.07) & (0.63, 0.91) & 0.0019 & 99.90\% & 0.83 & 0.83 (0.05) & (0.73, 0.94) & 0.0014\\
3-4 t/Week & 0.79 & 0.79 (0.07) & (0.66, 0.94) & 0.0022 & 99.70\% & 0.85 & 0.85 (0.05) & (0.74, 0.96) & 0.0016\\
\rowcolor{gray!6}  Daily & 0.90 & 0.90 (0.06) & (0.79, 1.03) & 0.0018 & 93.47\% & 0.93 & 0.93 (0.04) & (0.85, 1.02) & 0.0013\\
Never & - & - & - & - & - & - & - & - & \vphantom{1} -\\
\rowcolor{gray!6}  1-2 t/Mon & 0.98 & 0.98 (0.11) & (0.79, 1.19) & 0.0027 & 58.70\% & 0.99 & 0.99 (0.06) & (0.87, 1.11) & 0.0016\\
1-2 t/Week & 0.84 & 0.84 (0.08) & (0.70, 1.00) & 0.0022 & 97.31\% & 0.90 & 0.90 (0.05) & (0.81, 1.00) & 0.0014\\
\rowcolor{gray!6}  3-4 t/Week & 0.85 & 0.86 (0.08) & (0.71, 1.02) & 0.0021 & 96.05\% & 0.91 & 0.91 (0.05) & (0.82, 1.01) & 0.0013\\
Daily & 0.75 & 0.76 (0.05) & (0.66, 0.87) & 0.0016 & 100.00\% & 0.85 & 0.85 (0.04) & (0.78, 0.92) & 0.0011\\
\rowcolor{gray!6}  Never & - & - & - & - & - & - & - & - & -\\
1-2 t/Mon & 1.00 & 1.01 (0.12) & (0.81, 1.24) & 0.0041 & 50.61\% & 1.00 & 1.00 (0.07) & (0.88, 1.14) & 0.0029\\
\rowcolor{gray!6}  1-2 t/Week & 0.86 & 0.87 (0.09) & (0.72, 1.05) & 0.0036 & 93.73\% & 0.92 & 0.92 (0.06) & (0.82, 1.03) & 0.0024\\
3-4 t/Week & 0.89 & 0.90 (0.09) & (0.74, 1.08) & 0.0038 & 89.62\% & 0.93 & 0.94 (0.06) & (0.84, 1.05) & 0.0026\\
\rowcolor{gray!6}  Daily & 0.80 & 0.80 (0.07) & (0.69, 0.93) & 0.0031 & 99.04\% & 0.88 & 0.88 (0.05) & (0.81, 0.96) & 0.0020\\
\bottomrule
\multicolumn{10}{l}{\textit{Note: }}\\
\multicolumn{10}{l}{Abbreviations: SD, standard deviation; CrI, credible interval; MCSE, Monte Carlo Standard Error;}\\
\multicolumn{10}{l}{ Probability indicates that the p for HR smaller than 1.}\\
\end{tabular}
\end{table}

\begin{table}[H]
\caption{This is a table caption. Tables should be placed in the main text near to the first time they are cited.}
\centering
%% \tablesize{} %% You can specify the fontsize here, e.g.  \tablesize{\footnotesize}. If commented out \small will be used.
\begin{tabular}{ccc}
\toprule
\textbf{Title 1}    & \textbf{Title 2}  & \textbf{Title 3}\\
\midrule
entry 1     & data          & data\\
entry 2     & data          & data\\
\bottomrule
\end{tabular}
\end{table}

\hypertarget{discussion}{%
\section{Discussion}\label{discussion}}

\hypertarget{conclusion}{%
\section{Conclusion}\label{conclusion}}

% %%%%%%%%%%%%%%%%%%%%%%%%%%%%%%%%%%%%%%%%%%
% %% optional
% \supplementary{The following are available online at www.mdpi.com/link, Figure S1: title, Table S1: title, Video S1: title.}
%
% % Only for the journal Methods and Protocols:
% % If you wish to submit a video article, please do so with any other supplementary material.
% % \supplementary{The following are available at www.mdpi.com/link: Figure S1: title, Table S1: title, Video S1: title. A supporting video article is available at doi: link.}

\vspace{6pt}

%%%%%%%%%%%%%%%%%%%%%%%%%%%%%%%%%%%%%%%%%%
\acknowledgments{All sources of funding of the study should be
disclosed. Please clearly indicate grants that you have received in
support of your research work. Clearly state if you received funds for
covering the costs to publish in open access.}

%%%%%%%%%%%%%%%%%%%%%%%%%%%%%%%%%%%%%%%%%%
\authorcontributions{``X.X. and Y.Y. conceive and designed the
experiments; X.X. performed the experiments; X.X. and Y.Y. analyzed the
data; W.W. contributed reagents/materials/analysis tools; Y.Y. wrote the
paper.'\,'}

%%%%%%%%%%%%%%%%%%%%%%%%%%%%%%%%%%%%%%%%%%
\conflictsofinterest{The authors declare no conflict of interest. The
founding sponsors had no role in the design of the study; in the
collection, analyses, or interpretation of data; in the writing of the
manuscript, an in the decision to publish the results.}

%%%%%%%%%%%%%%%%%%%%%%%%%%%%%%%%%%%%%%%%%%
%% optional
\abbreviations{The following abbreviations are used in this manuscript:\\

\noindent
\begin{tabular}{@{}ll}
JACC & Japan Collaborative Cohort \\
FFQ & Food Frequency Questionnaire \\
\end{tabular}}


%%%%%%%%%%%%%%%%%%%%%%%%%%%%%%%%%%%%%%%%%%
% Citations and References in Supplementary files are permitted provided that they also appear in the reference list here.

%=====================================
% References, variant A: internal bibliography
%=====================================
%\reftitle{References}
%\begin{thebibliography}{999}
% Reference 1
%\bibitem[Author1(year)]{ref-journal}
%Author1, T. The title of the cited article. {\em Journal Abbreviation} {\bf 2008}, {\em 10}, 142--149.
% Reference 2
%\bibitem[Author2(year)]{ref-book}
%Author2, L. The title of the cited contribution. In {\em The Book Title}; Editor1, F., Editor2, A., Eds.; Publishing House: City, Country, 2007; pp. 32--58.
%\end{thebibliography}

% The following MDPI journals use author-date citation: Arts, Econometrics, Economies, Genealogy, Humanities, IJFS, JRFM, Laws, Religions, Risks, Social Sciences. For those journals, please follow the formatting guidelines on http://www.mdpi.com/authors/references
% To cite two works by the same author: \citeauthor{ref-journal-1a} (\citeyear{ref-journal-1a}, \citeyear{ref-journal-1b}). This produces: Whittaker (1967, 1975)
% To cite two works by the same author with specific pages: \citeauthor{ref-journal-3a} (\citeyear{ref-journal-3a}, p. 328; \citeyear{ref-journal-3b}, p.475). This produces: Wong (1999, p. 328; 2000, p. 475)

%=====================================
% References, variant B: external bibliography
%=====================================
\reftitle{References}
\externalbibliography{yes}
\bibliography{milk.bib}

%%%%%%%%%%%%%%%%%%%%%%%%%%%%%%%%%%%%%%%%%%
%% optional

%% for journal Sci
%\reviewreports{\\
%Reviewer 1 comments and authors’ response\\
%Reviewer 2 comments and authors’ response\\
%Reviewer 3 comments and authors’ response
%}

%%%%%%%%%%%%%%%%%%%%%%%%%%%%%%%%%%%%%%%%%%
\end{document}
